% Chapter Template

\chapter{An approach to VAD using PEFAC} % Main chapter title

\label{Chapter5} % Change X to a consecutive number; for referencing this chapter elsewhere, use \ref{ChapterX}

\lhead{Chapter 5. \emph{An approach to VAD using PEFAC}} % Change X to a consecutive number; this is for the header on each page - perhaps a shortened title

%----------------------------------------------------------------------------------------
%	SECTION 1 - PEFAC pitch tracker
%----------------------------------------------------------------------------------------

\section{PEFAC pitch tracker}

PEFAC pitch tracker \cite{PEFAC} is one of the recently published, robust pitch tracking algorithms which has been reported to achieve good results even at negative signal to noise ratios. Apart from calculating the pitch estimate as its primary output, it has also been designed to estimate the probability of a given frame \emph{being voiced} which is essentially a voiced speech activity detector. This value can be used to detect the voiced parts of a speech utterance by simple thresholding. Subsequent application of a hang-over scheme from Chapter \ref{Chapter3} should help in detecting the unvoiced segments. A combination of a thresholding stage and the hang-over scheme therefore creates a complete Voice Activity Detector.

The rest of this chapter is organised as follows. Firstly, an overview of PEFAC is presented. Then, the proposed approach to VAD is described in more detail, implemented and evaluated against LTSD, which was the best performing VAD from chapter \ref{Chapter4}. The experimental set-up for evaluation of PEFAC as VAD is the same as described in Chapter \ref{Chapter3}.

%-----------------------------------
%	SUBSECTION 1 - PEFAC algorithm description
%-----------------------------------

\subsection{PEFAC algorithm description}

A detailed description of PEFAC is available in \cite{PEFAC}. The algorithm can however be summarised in the following steps:
\begin{enumerate}
\item Transform the input signal to the power spectrum domain using short-time Fourier transform
\item Interpolate the periodogram of each frame onto a log-spaced frequency grid
\item Calculate the normalized periodogram
\item Convolve the normalized periodogram with a comb filter in order to enhance speech harmonics and attenuate the noise
\item Select the three highest peaks in the feasible range as the initial pitch candidates
\item Estimate the probability of a frame being voiced
\item Use dynamic programming to identify the final pitch estimate
\end{enumerate}

The probability of a frame being voiced is based on two features:
\begin{enumerate}
\item The log-mean power of a frame $L_t = \log E_t$ such that $E_t = \frac{1}{Q} \sum_{n=1}^{Q} Y_t(q_i)$ where $Q$ is the number of frequency bins, $Y_t(q)$ is the normalized log-frequency periodogram. Using the log-mean power is justified since voiced speech typically contains most energy contained in a complete speech utterance and its mean power is therefore higher than that of unvoiced speech
\item The ratio of the sum of the highest three peaks in the spectrum convolved with the comb filter to $E_t$. Voiced speech contains most of its power in the harmonic bins therefore using this measure as a second feature is justified
\end{enumerate}

%----------------------------------------------------------------------------------------
%	SECTION 2
%----------------------------------------------------------------------------------------

\section{Approach to VAD using PEFAC}

The first approach is relatively simple. PEFAC returns the probability of the current frame being voiced in a typical manner, i.e. as a number in the range 0 to 1. This can easily be thresholded and the voiced frames can be identified at this step. As stated numerously in the previous chapters, the voiced frames are typically surrounded by the unvoiced ones, hence application of the hang-over scheme from chapter \ref{Chapter3} will help to detect them. Overall, this approach should perform well in detection of both the voiced as well as unvoiced phonemes which is the aim of a complete Voice Activity Detector.

%----------------------------------------------------------------------------------------
%	SECTION 3
%----------------------------------------------------------------------------------------

\section{Evaluation results}

The proposed approach has been implemented, evaluated and compared with the LTSD VAD. In all experiments, the hang-over scheme has been applied to both algorithms. The evaluation of PEFAC without it does not make much sense, since there is no other way of detecting the unvoiced phonemes other than by the hang-over scheme. Other than that, the evaluation metrics are the same as in Chapter \ref{Chapter4}, namely the ROC curves and speech/non-speech hit rates with a fixed threshold.

The ROC curves for PEFAC and LTSD for all six noise types and two SNR levels (-5 dB and -10 dB) are presented in Figures ADD FIGURES. For the white, car and opsroom noises, the performance of both algorithms seems to be comparable. PEFAC clearly underperforms LTSD in the babble and spchspect noises, which is expected to some extent, since being primarily a pitch tracker, the babble is probably the most difficult noise in which such algorithm has to operate. Interestingly, PEFAC is far superior to LTSD in the factory noise and its performance is much better in both SNR levels.

